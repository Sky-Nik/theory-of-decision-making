% cd ..\..\Users\NikitaSkybytskyi\Desktop\c3s2\tpr
\documentclass[a4paper, 12pt]{article}
\usepackage[utf8]{inputenc}
\usepackage[english, ukrainian]{babel}

\usepackage{amsmath, amssymb}
\usepackage{multicol}
\usepackage{graphicx}
\usepackage{float}
\usepackage{multicol}

\allowdisplaybreaks
\setlength\parindent{0pt}
\numberwithin{equation}{subsection}

\usepackage{hyperref}
\hypersetup{unicode=true,colorlinks=true,linktoc=all,linkcolor=red}

\numberwithin{equation}{section}% reset equation counter for sections
\numberwithin{equation}{subsection}
% Omit `.0` in equation numbers for non-existent subsections.
\renewcommand*{\theequation}{%
  \ifnum\value{subsection}=0 %
    \thesection
  \else
    \thesubsection
  \fi
  .\arabic{equation}%
}

\usepackage{fancyhdr}
\pagestyle{fancy}
\setlength{\headheight}{15pt}
\rhead{Нікіта Скибицький, ОМ-3}

\usepackage{pagecolor,xcolor}
\color{yellow}
\begin{document}
\pagecolor{black}

\tableofcontents

\section{\S3. Функції корисності в умовах ризику та невизначеності}

Найважливішим застосуванням теорії очікуваної корисності є можливість формалізації процесу прийняття рішень в умовах ризику й невизначеності. \medskip

Задача прийняття рішень (ЗПР) є визначеною на наступній тріаді множин: $X$ --- множина альтернатив; $Y$ --- множина наслідків; $S$ --- множина станів.  \medskip

Множина $S$ є проявом стохастичної невизначеності у прийнятті рішень, причому конкретна інтерпретація станів залежить від формулювання задачі (наприклад, попит на ту чи іншу продукцію, погода і т. п.). Множину $S$ також називають множиною ``станів природи'' чи ``станів зовнішнього середовища'', щоб підкреслити властиву їй невизначеність і незалежність від ОПР. \medskip

Розглянемо конкретні види функцій корисності (критеріїв) для нормальної форми ЗПР, які найчастіше вживаються в методах прийняття рішень.

\subsection{Мінімаксний критерій Вальда}

Мінімаксний критерій (ММ) використовує функцію корисності альтернатив MM 
\[E_{MM}(x) = \min\limits_{s \in S} f(x, s),\]

що відповідає позиції крайньої обережності. Шукана альтернатива вибирається з умови 
\[x^\star \in \arg \max\limits_{x \in X} E_{MM}(x) = \arg \max\limits_{x \in X} \min\limits_{s \in S} f(x, s).\]

\subsection{Критерій Байєса-Лапласа}

На відміну від мінімаксного критерію, цей критерій враховує кожен із можливих наслідків альтернативи. Нехай $p(s)$ --- імовірність появи стану $s \in S$, тоді для BL-критерію корисність кожної альтернативи характеризується математичним сподіванням корисностей її наслідків
\[E_{BL}(x) = \int\limits_{s \in S} p(s) f(x, s) \, \mathrm{d} s.\]

Шукана альтернатива вибирається з умови:
\[x^\star \in \arg \max\limits_{x \in X} \int\limits_{s \in S} p(s) f(x, s) \, \mathrm{d} s = \arg \max\limits_{x \in X} \int\limits_{s \in S} p(s) f(x, s) \, \mathrm{d} s.\]

\subsection{Критерій мінімізації дисперсії оцінки}

Цей критерій використовують, коли ОПР, зацікавлена в отриманні ``стійкого'' щодо станів середовища рішення і відомо, що ймовірності станів середовища мають нормальний розподіл. При виборі цього критерію кожна альтернатива оцінюється дисперсією функції корисності її наслідків при всіх мінімізованих станах середовища:
\[E_D(x) = \int\limits_{s \in S} p(s) \left( \int\limits_{s' \in S} p(s') f(x, s') \, \mathrm{d} s' - f(x, s) \right)^2 \mathrm{d} s = \int\limits_{s \in S} p(s) (E_{MM}(x) - f(x, s))^2 \, \mathrm{d} s.\]

У свою чергу шукана альтернатива вибирається з умови
\[x^\star \in \arg\min\limits_{x \in x} E_D(x) = \arg\min\limits_{x \in x} \int\limits_{s \in S} p(s) \left( \int\limits_{s' \in S} p(s') f(x, s') \, \mathrm{d} s' - f(x, s) \right)^2 \mathrm{d} s.\]

\subsection{Критерій максимізації ймовірності}

При використанні цього критерію ОПР фіксує величину оцінки функції корисності наслідків 
\[f_0: \min\limits_{x \in X} \min\limits_{s \in S} f(x, s) \le f_0 \le \max\limits_{x \in X} \max\limits_{s \in S} f(x, s),\]
яку він хоче ймовірно досягти. \medskip

Для кожної альтернативи $x$ визначається ймовірність $\mathbb{P} \{ f(x, s) \ge f_0 \}$ того, що функція корисності наслідків буде не менша за $f_0$ для кожного стану середовища $s \in S$. Критерій полягає в максимізації ймовірності досягнення значення заданої оцінки
\[E_F(x) = \int\limits_{s \in S, f(x, s) \ge f_0} p(s) \, \mathrm{d} s,\]
а шукана альтернатива вибирається з умови
\[x^\star \in \arg\max\limits_{x \in X} E_F(x) = \arg\max\limits_{x \in X} \int\limits_{s \in S, f(x, s) \ge f_0} p(s) \, \mathrm{d} s.\]

\subsection{Модальний критерій}

Суть цього критерію полягає у виборі альтернативи, виходячи з найбільш імовірного стану середовища 
\[s^\star \in S: s^\star = \arg\max\limits_{s \in S} p(s).\]

При використанні цього критерію ОПР вважає, що середовище знаходиться у стані $S^\star$ і вибирає альтернативу з критерію 
\[E_{MOD}(x) = \max\limits_{x \in X} f(x, s^\star),\]
тобто з умови
\[x^\star \in \arg\max\limits_{x \in X} f(x, \arg\max\limits_{s \in S} p(s)).\]

\subsection{Критерій Севіджа ($S$-критерій)}

За цим критерієм корисність кожної альтернативи характеризується
\[E_{SE}(x) = \max\limits_{s \in S} \left( \max\limits_{x' \in X} f(x', s) - f(x, s) \right).\]

Цю величину можна інтерпретувати як втрати (штрафи), що виникають у стані $s \in S$ при заміні оптимальної для неї альтернативи на альтернативу $x$. Тоді логічно приймати рішення за умовою мінімізації максимально можливих втрат:
\[x^\star \in \arg\min\limits_{x \in X} \max\limits_{s \in S} \left( \max\limits_{x' \in X} f(x', s) - f(x, s) \right).\]

\subsection{Критерій стабільності ($V$-критерій)}

Із метою мати ``максимальну незалежність від станів природи'' (``мінімальну залежність'') О. Волошин запропонував такий критерій ``стабільності''. За цим критерієм корисність кожної альтернативи характеризується величиною
\[E_{ST}(x) = \max\limits_{s \in S} \left( \max\limits_{s' \in S} f(x, s') - f(x, s) \right).\]

Цю величину можна інтерпретувати як втрати, що виникають при виборі альтернативи $x \in X$, при реалізації стану природи $s \in S$. Як оптимальну альтернативу логічно прийняти ту, для якої різниця між максимальним і мінімальним виграшами буде мінімальною, тобто розв'язок задачі прийняття рішень за V-критерієм вибирається з множини:
\[x^\star \in \arg\min\limits_{x \in X} E_{ST}(x) = \arg\min\limits_{x \in X} \max\limits_{s \in S} \left( \max\limits_{s' \in S} f(x, s') - f(x, s) \right).\]

Якщо розв'язок не єдиний, його необхідно вибирати з недомінованих альтернатив.

\subsection{Критерій Гурвіца}

Намагаючись зайняти найбільш урівноважену позицію, Л. Гурвіц запропонував критерій GW, функція корисності якого забезпечує компроміс між граничним оптимізмом і крайнім песимізмом. За цим критерієм корисність кожної альтернативи характеризується величиною 
\[E_{GW}(x) = \alpha \cdot \max\limits_{s \in S} f(x, s) + (1 - \alpha) \cdot \min\limits_{s \in S} f(x, s),\]
де $\alpha \in [0, 1]$ --- ваговий коефіцієнт, що характеризує схильність ОПР до ризику. \medskip

Рішення приймається з умови:
\[x^\star \in \arg\max\limits_{x \in X} E_{GW}(x) = \arg\max\limits_{x \in X} \left( \alpha \cdot \max\limits_{s \in S} f(x, s) + (1 - \alpha) \cdot \min\limits_{s \in S} f(x, s) \right).\]

Для $\alpha = 0$ GW-критерій перетворюється в МM-критерій. Для $\alpha = 1$ він перетворюється у критерій азартного гравця. На практиці вибрати цей коефіцієнт буває так само важко, як правильно вибрати сам критерій. Навряд чи можливо знайти кількісну характеристику для тих часток оптимізму й песимізму, що присутні при прийнятті рішення. Тому найчастіше $\alpha = 0.5$ без заперечень приймається як деякої ``середньої'' точки зору. \medskip

Інколи величина $\alpha$ використовується для обґрунтування вже прийнятого рішення. Для рішення, що сподобалося, обчислюється ваговий коефіцієнт $\alpha$ і він інтерпретується як показник співвідношення оптимізму та песимізму. Таким чином, позиції, виходячи з яких приймаються рішення, можна розсортувати принаймні заднім числом.

\subsection{Критерій Ходжа-Лемана}

Цей критерій спирається одночасно на ММ-критерій і ВL-критерій. Функція корисності альтернатив визначається як
\[E_{HL}(x) = \alpha \int\limits_{s \in S} p(s) f(x, s) \, \mathrm{d} s + (1 - \alpha) \cdot \min\limits_{s \in S} f(x, s).\]

Для $\alpha = 0$ HL-критерій перетворюється в МM-критерій, а для $\alpha = 1$ він перетворюється в BL-критерій. Ступінь впевненості $\alpha \in [0, 1]$ в будь-якому розподілі ймовірностей $p(s)$, $s \in S$, практично не піддається оцінці. Таким чином, вибір параметра $\alpha$ є повністю суб'єктивним.

\subsection{Критерій Гермейєра}

За підходом Ю. Гермейєра до відшукання слабко ефективних рішень у задачах багатокритеріальної оптимізації можна запропонувати ще один критерій (GE). \medskip

Нехай множина станів є скінченною, а саме $S = \{s_1, \ldots, s_n\}$. Не обмежуючи загальності, будемо вважати $f(x, s) < 0$ ($f(x, s)$ інтерпретуються як витрати) $\forall x \in X$, $\forall s \in S$, тоді функція корисності альтернатив за GE-критерієм визначається як 
\[E_{GE}(x) = \min\limits_{s \in S} p(s) f(x, s),\]
а рішення приймається з умови:
\[x^\star \in \arg\max\limits_{x \in X} E_{GE}(x) = \arg\max\limits_{x \in X} \min\limits_{s \in S} p(s) f(x, s),\]

Імовірності станів природи $p(s)$, $s \in S$, у цьому критерії можна інтерпретувати як вагові коефіцієнти функцій корисності $f(x, s)$, $s \in S$, наслідків, які хочемо одночасно максимізувати. 

\subsection{Критерій добутків}

Цей критерій базується на ідеї фільтрації інформації, яка застосовується в теорії нечітких множин. Добутком функцій належності нечітких множин визначається одна з операцій перетину нечітких множин. \medskip

Нехай множина станів є скінченною, а саме $S = \{s_1, \ldots, s_n\}$. Не обмежуючи загальності, будемо вважати $f(x, s) > 0$, $\forall x \in X$, $\forall s \in S$. \medskip

Критерій добутків МU використовує функцію корисності альтернатив 
\[E_{MU}(x) = \prod\limits_{s \in S} f(x, s).\]

Шукана альтернатива вибирається з умови:
\[x^\star \in \arg\max\limits_{x \in X} E_{MU}(x) = \arg\max\limits_{x \in X} \prod\limits_{s \in S} f(x, s).\]
\end{document}