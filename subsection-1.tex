\subsection{Загальна задача прийняття рішень}

Схему прийняття рішень можна описати в такому вигляді (див. рис. \ref{fig:1.1.1}). \\

Загалом кожен блок 1–5 наведеної схеми (``загальної задачі прийняття рішень'' -- ЗЗПР) потребує конкретизації та певної формалізації. Задача із заданою множиною альтернатив $\Omega$ і принципом оптимальності ОП називається загальною задачею оптимізації, зміст якої полягає у виділенні множини ``кращих'' альтернатив ОП($\Omega$), зокрема, якщо принцип оптимальності задається скалярною функцією вибору на $\Omega$, то маємо звичайну \textit{оптимізаційну задачу} (наприклад, лінійного програмування). Якщо принцип оптимальності задається множиною \textit{критеріальних функцій}, то маємо задачу \textit{багатокритеріальної оптимізації}. Задача з відомою множиною альтернатив $\Omega$ і явно заданим принципом оптимальності називається \textit{задачею вибору}.

\begin{figure}
	\label{fig:1.1.1}
	\begin{enumerate}
		\item Визначення мети (цілі) та засобів її досягнення
		\item Побудова варіантів досягнення мети (множини альтернатив)
		\item Формування множини наслідків (оцінка альтернатив)
		\item Визначення принципу порівняння альтернатив (визначення принципу оптимальності)
		\item Структурування множини альтернатив:
		\begin{enumerate}
			\item розбиття на класи (кластеризація);
			\item упорядкування;
			\item вибір кращої (кращих).
		\end{enumerate}
	\end{enumerate}
\end{figure}

У процесі розв'язання загальної задачі прийняття рішень, як правило, беруть участь три групи осіб: \textit{особи, що приймають рішення} (ОПР), \textit{експерти} (Е) та \textit{консультанти} (К). \\

ОПР називають людину (або колективний орган такий, як науковий заклад, Верховна рада), що має (формує) ціль, яка слугує мотивом постановки задачі та пошуку її розв'язання. ОПР визначає також, які засоби є допустимими (недопустимими) для досягнення мети. \\

\textit{Експерт} -- це спеціаліст у своїй галузі, що володіє інформацією про задачу, але не несе прямої відповідальності за результати її розв'язання. Експерти допомагають ОПР на всіх стадіях постановки й розв'язання ЗПР. \\

\textit{Аналітиками} (консультантами, дослідниками) називають спеціалістів із теорії прийняття рішень. Вони розробляють модель (математичну, інформаційну) задачі прийняття рішень (ЗПР), процедури прийняття рішень, організовують роботу ОПР і експертів. \\

У найпростіших ситуаціях ОПР може виступати одним у трьох ролях, у більш складних -- ОПР може поєднувати функції аналітика, звертаючись до спеціалістів із вузьким профілем для вирішення часткових проблем. У загальному випадку ОПР (наприклад, президент або профільний комітет Верховної Ради) залучає до вирішення державних проблем аналітиків -- консультантів, які, у свою чергу, звертаються до експертів. ОПР -- головнокомандуючий має колективного консультанта -- Генеральний штаб, який, у свою чергу, організовує роботу експертів -- спеціалістів з озброєння, хімічного й біологічного захисту, політологів, метеорологів тощо. \\

У практичних (прикладних) задачах прийняття рішень формалізація кожного кроку процесу прийняття рішень (поданих на рис. \ref{fig:1.1.1}) пов'язана з певними, іноді дуже складними, проблемами. У першу чергу, постає проблема визначення мети та засобів її досягнення. Можна ставити апріорі недосяжні або навіть абсурдні чи злочинні цілі (``моя мета -- пробігти стометрівку за п'ять секунд'', ``наша мета -- комунізм'', ``наша мета -- чистота раси'' і т. д.). Можна використовувати нецивілізовані, навіть злочинні, методи досягнення цілком досяжної мети (``мета -- стати президентом'', ``стати багатим'', ``отримати п'ятірку на іспиті'' і т. д.). \\

Але зараз не про це. Нас цікавить формалізація ЗЗПР, її описання на мові математики з метою моделювання практичних ситуацій прийняття рішень. І якщо математична модель призведе до факту неіснування розв'язку поставленої задачі, наша мета буде досягнутою. Припустимо тепер, що мета й методи її досягнення визначені. Постає проблема побудови множини альтернатив -- варіантів дій, направлених на досягнення мети. Тут, у першу чергу, виникає проблема побудови ``повного списку'' альтернатив. Можлива ситуація, коли не включення певної альтернативи призведе до неможливості розв'язання задачі або до її ``неякісного'' розв'язання. Так, не включення до економічної системи колишнього СРСР ринкових механізмів призвело до деградації суспільства й розпаду СРСР. \\

Не менш складною є проблема оцінки альтернатив -- ``до чого приведе та чи інша вибрана дія''. Як правило, оцінки альтернатив мають суб'єктивний характер, вони отримуються на основі обробки експертної інформації. Навіть якщо можна оцінювати альтернативи за допомогою ``об'єктивних'' процедур (наприклад, вимірювати вагу товару, відстань між населеними пунктами), постає проблема визначення всіх або хоча б ``найважливіших'' аспектів оцінки кожної альтернативи. Тут також неврахування навіть одного аспекту в оцінці варіантів дії може призвести до катастрофічних наслідків (згадаймо Чорнобильську катастрофу -- неврахування ризику аварії при будівництві
АЕС привело до трагічних результатів). \\

Остання принципова складність (але не остання за значенням) -- вибір принципу порівняння альтернатив і на його основі -- принципу оптимальності. Якщо на попередньому етапі визначені числові оцінки альтернатив, то вибір принципу оптимальності зводиться до вибору критерію (критеріїв) оптимізації, який максимально відповідає меті ЗЗПР. Так, якщо для тренера футбольної команди мета -- перемога у наступному матчі, то за принцип оптимальності може слугувати такий критерій: ``Перемагає та команда, яка виконує за матч більшу сумарну кількість успішних тактично-технічних елементів'' (передач м'яча, відборів, ударів по воротах і т. д.). Такий принцип не раз висловлював В. Лобановський. Як правило, визначення (побудова, прийняття) принципу оптимальності відбувається у декілька етапів. Так, якщо мета ЗЗПР описується декількома числовими критеріями (і, отже, маємо задачу багатокритеріальної оптимізації), необхідно визначити -- на основі якого ``глобального'' принципу оптимальності будуть порівнюватись (і вибиратись кращі) альтернативи. \\

Проблеми реалізації останнього блоку схеми пов'язані, у першу чергу, із математичними труднощами розв'язання задач, що виникають. Тут і велика розмірність, і проблеми існування розв'язку, і збіжність процедур його побудови і т. д. \\

Розглянемо приклади змістовної інтерпретації блоків ЗЗПР.
\begin{enumerate}
	\item \textbf{Визначення мети та засобів.} Розглянемо такі приклади.
	\begin{enumerate}
		\item Християнська доктрина визначає мету земного існування людини як ``спасіння душі''. Засоби -- будь-які, що не суперечать заповідям ``Нового заповіту'' (не вбивай; не гнівайся на ближнього; не чини перелюбу; не клянись, але виконуй клятви свої перед Господом; не протився злому, і коли вдарить тебе, хто у праву щоку твою, -- підстав йому й іншу; любіть і ворогів своїх; про милостиню; про піст; складайте собі скарби на небі; покладайте на Бога надію свою; не судіть своїх ближніх (``Не суди, але викривай''); ходіть дорогою вузькою; стережіться фальшивих пророків; чиніть волю отця вашого небесного; не будуйте на піску).
		\item Сучасна гуманістична доктрина визначає ціль життя людини як ``самореалізацію'' (Е. Фромм) [44]. Засоби досягнення цієї мети визначаються, перш за все, ``Декларацією прав людини'', у якій на першому місці, безумовно, стоїть християнський принцип ``не убий'' (``право на життя''). Інші біблійні принципи не є категоричними імперативами. Із принципом ``Не вбивай'' тісно пов'язана проблема смертної кари. Якщо мета -- ``справедливість за будь-яку ціну'' (зокрема, ``око за око, зуб за зуб''), то смертна кара допустима. Але, якщо дещо переформулювати проблему смертної кари -- чи згодні ви, щоб разом із шістьма злочинцями був страчений один невинний (а саме така статистика хибних смертних вироків за останні 150 років у Європі й Америці), то принцип ``справедливість за будь-яку ціну'' стає зовсім не очевидним.
		\item Видатний філософ ХХ ст. Микола Бердяєв визначав мету життя людини не як ``спасіння'', а як ``творче сходження'', засіб -- ``свобода'' [13].
		\item ``Хто ж вони, справжні філософи? Ті, хто метою мають істину'' (Платон). \\
		``Життя перестає прив'язувати до себе щойно зникає мета'' (І. Павлов). \\ 
		``Минуле і сучасне -- наші засоби, тільки майбутнє -- наша мета'' (Блез Паскаль). \\
		``Мета влади -- влада'' (Дж. Оруел).
		\item Мета -- вилікувати хворого. Засоби -- усе те, що надається системою охорони здоров'я.
		\item Мета -- побудувати літак. Засоби -- 300 млн грн. на початок 2010 р. 
		\item Мета -- виграти футбольний матч, засоби у тренера -- сформувати команду на даний матч із наявних 25 футболістів. 
		\item Мета -- ``щастя всього людства''. Цю мету висували й висувають філософи, політики, пророки, авантюристи. І якщо з метою все зрозуміло (про формалізацію терміна ``щастя'' тепер не йдеться -- на Всесвітньому економічному форумі в Давосі в січні 2006 р. один із семінарів мав назву ``Щастя -- це...''), то із засобами її досягнення набагато складніше. Згадаймо хоча б Ф. Достоєвського -- ``щастя всього людства не варте однієї сльозинки дитини''; Мао Цзедуна -- ``заради перемоги соціалізму можна пожертвувати половиною людства''; Ф. Ніцше -- ``хочеш бути щасливим -- не мрій''. \\

		Доцільно тут згадати і слова: ``Політики -- це люди, найбільш нерозбірливі в засобах (досягнення мети)''. Сучасна історія, на жаль, повністю підтверджує цей вислів. Прикладів тут безліч, і читач легко може їх навести. Ми ж лише процитуємо слова Максима Горького про те, що ``Леніну, як вождю, притаманна для цієї ролі необхідність у відсутності моралі'' і слова Мітчела Канора (розробника ``Lotus''), який називає Білла Гейтса ``найуспішнішим і яскравим представником тих, хто грає на стратегії -- перемога за будь-яку ціну''. А взагалі проблема ``мета -- засоби'' стара як світ. Ми ж приєднуємось до думки, що історичний досвід показує, що відмова від вимог моралі є завжди програшною стратегією.
		\item Мета -- отримання максимального задоволення від життя (про формалізацію поняття задоволення див. вище у Е. Фромма). Засоби студента -- 50 грн (на початок 2009 р.).
		\item Мета викладача -- навчити студента своєму предмету, засоби викладання -- ``цікаво, зрозуміло і ... весело'' (принцип видатного вченого і педагога ХХ ст. академіка -- фізика П. Капіци).
	\end{enumerate}
	Загальний підхід до поняття мети був розвинутий на початку 40-х рр. ХХ ст., у першу чергу, Н. Вінером, який писав, що термін ``цілеспрямоване'' означає, що дія або поведінка допускає тлумачення як направлені на досягнення деякої мети, тобто деякого кінцевого стану, при якому об'єкт вступає у певний зв'язок у просторі або часі з деякими іншими об'єктами або подіями. Із філософськими аспектами в zоб'єктивізації поняття мети можна ознайомитись у роботі [11]. \\

	Розглянемо основні типи цілей і способи їхньої формалізації, що застосовуються при прийнятті рішень. \\

	\textit{``Якісна''} ціль характеризується тим, що будь-який результат або повністю задовольняє ці цілі або повністю не задовольняє, причому результати, що задовольняють ці цілі нерозрізненні між собою точно так як нерозрізнені між собою й результати, що не задовольняють ці цілі. Наприклад, ціль -- стати чемпіоном. І якщо ціль досягнуто, то немає значення, як її досягнуто -- наполегливим тренуванням, підкупом суддів, знищенням конкурентів тощо. Якісну ціль можна формалізувати у вигляді деякої підмножини $A$ множини всіх можливих результатів, де будь-який результат $a \in A$ задовольняє цій цілі, а будь-який результат $a \notin A$ не задовольняє їй. Множина А при цьому називається цільовою підмножиною. Так, якщо ціль ``зайняти призове місце'', то цільова множина $A$ -- перші три місця з усіх можливих. \\

	Якісну ціль (її можна назвати якісною ``чіткою'' ціллю) можна узагальнити так. Нехай кожному результату а відповідає ``ступінь'' виконання цілі $\mu(a)$, $0 \le \mu(a) \le 1$. Зокрема, якщо ціль чітка, то $\mu(a) = 1$, якщо $a \in A$; $\mu(a) = 0$, якщо $a \notin A$. Так, нехай у останньому прикладі $\mu(\text{I місце}) = 1$, $\mu(\text{II}) = 0.9$, $\mu(\text{III}) = 0.7$ і ``нечітка'' цільова множина А визначається умовою: $\mu(a) \ge 0.7$. Зазначимо, що значення функції $\mu(a)$ (функція ``належності'' -- див. Розділ 7 ``Прийняття рішень в умовах нечіткої інформації'') не є оцінкою результату $a$, а лише ``ступінь'' його належності до цільової множини $A$ (можливо ``націлюватись'' на І місце недоцільно з погляду прикладання надмірних зусиль). \\

	\textit{``Кількісна''} ціль є результатом вибору на множині результатів, що описуються кількісно, за допомогою деякої дійснозначної функції $f: A \to E^1$. Задача прийняття рішень у цьому випадку зводиться до знаходження оптимуму (максимуму чи мінімуму) функції $f$ на множині $А$. Зазначимо, якщо ``якісну чітку'' ціль формально можна звести до ``кількісної'' (поклавши, наприклад, $f(a) = 1$, $a \in A$; $f(a) = 0$, $a \notin A$), то ЗПР з ``якісною нечіткою'' ціллю вимагають додаткової інформації до своєї формалізації. \\

	Якщо цільова функція є векторною, тобто кожен результат описується набором чисел, що характеризують його ``вартість'', ``ефективність'', ``надійність'', то маємо задачу багатокритеріальної оптимізації. \\

	Зазначимо, якщо ціль задано з допомогою скалярної цільової функції $f$, то можна визначити пов'язану з цією ціллю перевагу серед результатів: із двох результатів кращим буде той, якому відповідає більше (менше) значення цільової функції (при рівних значеннях цільової функції говорять про байдужність результатів). Назвемо таку перевагу перевагою, що пов'язана з цільовою функцією $f$. Але можна говорити про перевагу й без наявності цільової функції, задаючи множину пар результатів, для яких перший результат у парі є кращим за другий (або не гіршим). Останнє означає, що на декартовому добутку результатів $A \times A$ задане деяке бінарне відношення. За заданим бінарним відношенням у загальному випадку неможливо побудувати цільову функцію, пов'язану з ним. Відомі достатні умови (властивості), яким повинно задовольняти бінарне відношення для того, щоб існувала цільова функція, пов'язана з ним (див. Розділ 2 ``Основи теорії корисності''). Отже, задання переваг у вигляді бінарного відношення на множині результатів є більш загальною формою формалізації цілі. З іншого боку, на практиці дуже часто відношення переваги задається саме бінарним порівнянням -- про це говорить і народна мудрість ``Усе пізнається у порівнянні''.
\end{enumerate}