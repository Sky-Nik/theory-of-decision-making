\section{Вступ}

Сподіваємось ніхто не буде заперечувати проти такого визначення: ``життя -- це процес прийняття рішень''. Рішення приймають політики, військові, виробники, споживачі, продавці, покупці, водії, пішоходи (``йти чи не йти на червоне світло''), дорослі (``що робити з дітьми''), діти (``що робити з іграшкою''), рішення приймають навіть студенти (``йти чи не йти на лекцію'', а якщо йти, то що робити -- слухати лектора, розмовляти з сусідом). Рішення приймаються колективні (вибори президента), індивідуальні (за якого кандидата голосувати), стратегічні (``куди піти вчитися''), тактичні (``брати чи не брати з собою парасольку''), миттєві (воротар -- ``у який кут стрибати''), розтягнуті в часі та просторі, важливі (з погляду цивілізації, партії, окремого індивіда), несуттєві (``яку програму по телевізору дивитися'') і т. п. Рішення приймаються на основі знань, досвіду, інтуїтивно, за допомогою випадкового механізму, за підказкою інших, за бажанням, за необхідністю. У багатьох практично цікавих випадках основним моментом є саме метод (алгоритм) прийняття рішення, а вже потім вивчення властивостей прийнятого рішення. Більше того, у деяких випадках апріорне задання властивостей шуканого рішення (у вигляді аксіом) призводить до його неіснування або до неможливості його знаходження заданою процедурою. \\

Хоча вивченням окремих задач прийняття рішень людство займалось давно, теорія прийняття рішень як наукова дисципліна сформувалось у другій половині ХХ ст., що пов'язано, у першу чергу, з розвитком обчислювальної техніки й інформатики. \\

Термін ``прийняття рішень'' зустрічається в багатьох дисциплінах, прийняття рішень є одним з основних напрямів прикладної математики. Моделі та методи теорії прийняття рішень знайшли широке застосування, у першу чергу, в економіці, військовій справі, політиці, медицині. Історично теорія прийняття рішень виокремилась із наукового напряму, відомого під назвою ``дослідження операцій''. У свою чергу, теорія прийняття рішень стимулювала розвиток нового наукового напряму ``штучний інтелект''. \\

Таким чином, із погляду навчального плану напряму ``прикладна математика та інформатика'' теорія прийняття рішень є проміжною ланкою між дисциплінами ``дослідження операцій'' (``методи оптимізації'') та ``штучний інтелект'' (``проектування баз знань'').